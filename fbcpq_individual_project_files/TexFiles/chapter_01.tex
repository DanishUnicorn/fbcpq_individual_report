%\chapter*{Preface and abstract}
\setlength{\headheight}{12.71342pt}
\addtolength{\topmargin}{-0.71342pt}


\section*{Preface}
This written report has been prepared as part of the course NPLK14014U - Fruit and Berry Crop Physiology and Quality at the University of Copenhagen. The course explores the physiological and biochemical factors influencing fruit and berry quality, including aspects related to nutrition, flavour, and post-harvest characteristics.

\vspace{1em}
The present project is a theoretical study on the development and composition of aroma in fruits, with a particular focus on the biochemical formation of volatile compounds and their relationship to ripening and quality. The objective of this report is to synthesise knowledge from scientific literature and course materials to understand the factors influencing aroma development, including enzymatic pathways, environmental influences, and post-harvest changes.

\vspace{1em}
The assignment was carried out by Lucas Daniel Paz Zuleta (TZS159), MSc student at the University of Copenhagen, as an individual written report fulfilling the course requirements.

\section*{Abstract\textsuperscript{4}}
This project investigates the theoretical potential of a hemp seed protein bar as a sustainable alternative to other protein bars on the market. The aim of the project was to design a nutrient-rich product with a favourable environmental profile, while addressing market demands for plant-based and health-oriented food. The nutritional composition was assessed through literature-based data on macronutrients, i.e. protein, dietary fibres, and fatty acids, with focus on the protein profile. Comparisons were made to existing market products (ROO'bar hemp protein bar), highlighting the bar's potential for high protein. Through comparing EFSA threshold to product composition, it was investigated if the product could opbtain claims as "high protein" and "high fibre". The lipid fraction showed a desirable omega-6 to omega-3 ratio, although thresholds for authorised health claims were not reached. Environmental perspectives further emphasised the advantages of hemp cultivation, including low carbon footprint, soil health benefits, and potential use of side streams. Overall, hemp seeds shows potential in developing innovative plant-based products that align with both nutritional and sustainability goals.


\vspace{2em}
\begin{center}
    \textbf{Signatures}\\[0.5em]
    {\small Copenhagen, \today}
    \end{center}
    
    \vspace{2.5em}
    
    \noindent
    \begin{minipage}[t]{0.48\textwidth}
      \stackengine{0pt}{\raisebox{-1.2em}{\includegraphics[width=0.7\linewidth]{KU_titelpage/signatures/lucas_daniel_paz_zuleta.png}}}{\rule{\linewidth}{0.4pt}}{O}{c}{F}{F}{L}\\[0.5em]
      Lucas Daniel Paz Zuleta (TZS159)\\
      {\small \today}
  \end{minipage}\hfill
    %\begin{minipage}[t]{0.48\textwidth}
    %  \rule{\linewidth}{0.4pt}\\[-0.2em]
    %  Nils Hugo Nilsson (XQK212)\\
    %  {\small \today}
    %\end{minipage}
    
    \vspace{7em}
    
    \noindent
    %\begin{minipage}[t]{0.48\textwidth}
    %  \rule{\linewidth}{0.4pt}\\[-0.2em]
    %  Niclas Hauerberg Hyldahl (JNC117)\\
    %  {\small \today}
    %\end{minipage}\hfill
    %\begin{minipage}[t]{0.48\textwidth}
    %  \rule{\linewidth}{0.4pt}\\[-0.2em]
    %  Lucas Daniel Paz Zuleta (TZS159)\\
    %  {\small \today}
    %\end{minipage}