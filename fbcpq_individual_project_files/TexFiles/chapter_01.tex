%\chapter*{Preface and abstract}
\setlength{\headheight}{12.71342pt}
\addtolength{\topmargin}{-0.71342pt}


\section*{Preface\textsuperscript{4}}
This written assignment has been prepared as part of the course NPLK14014U - Fruit and Berry Crop Physiology and Quality at the University of Copenhagen. %The course addresses the challenges of developing sustainable and healthy diets by examining processing effects on nutrients in dairy products, hybrid products and their alternatives. 

\vspace{1em}
The project is a theoretical study on the %development of a hemp seed protein bar as a dairy substitute. Through this work, we aimed to apply the knowledge and competences obtained during the course, including nutritional evaluation, processing considerations, sustainability aspects, and consumer perspectives. The assignment was carried out by Sofie Karoline Thue Hansen (FVC568), Nils Hugo Nilsson (XQK212), Niclas Hauerberg Hyldahl (JNC117), and Lucas Daniel Paz Zuleta (TZS159), all MSc students at the University of Copenhagen.

\vspace{1em}
Although the project was written by all group members, each section was primarily authored by one to two members, as indicated in the superscript in the section headings. The superscript will be numbered from 1-4, depending on the main author. 1 = Sofie, 2 = Hugo, 3 = Niclas, and 4 = Lucas. The report has been reviewed and edited collectively to ensure coherence and quality.

\section*{Abstract\textsuperscript{4}}
This project investigates the theoretical potential of a hemp seed protein bar as a sustainable alternative to other protein bars on the market. The aim of the project was to design a nutrient-rich product with a favourable environmental profile, while addressing market demands for plant-based and health-oriented food. The nutritional composition was assessed through literature-based data on macronutrients, i.e. protein, dietary fibres, and fatty acids, with focus on the protein profile. Comparisons were made to existing market products (ROO'bar hemp protein bar), highlighting the bar's potential for high protein. Through comparing EFSA threshold to product composition, it was investigated if the product could opbtain claims as "high protein" and "high fibre". The lipid fraction showed a desirable omega-6 to omega-3 ratio, although thresholds for authorised health claims were not reached. Environmental perspectives further emphasised the advantages of hemp cultivation, including low carbon footprint, soil health benefits, and potential use of side streams. Overall, hemp seeds shows potential in developing innovative plant-based products that align with both nutritional and sustainability goals.


\vspace{2em}
\begin{center}
    \textbf{Signatures}\\[0.5em]
    {\small Copenhagen, \today}
    \end{center}
    
    \vspace{2.5em}
    
    \noindent
    \begin{minipage}[t]{0.48\textwidth}
      \rule{\linewidth}{0.4pt}\\[-0.2em]
      Lucas Daniel Paz Zuleta (TZS159)\\
      {\small \today}
    \end{minipage}\hfill
    %\begin{minipage}[t]{0.48\textwidth}
    %  \rule{\linewidth}{0.4pt}\\[-0.2em]
    %  Nils Hugo Nilsson (XQK212)\\
    %  {\small \today}
    %\end{minipage}
    
    \vspace{7em}
    
    \noindent
    %\begin{minipage}[t]{0.48\textwidth}
    %  \rule{\linewidth}{0.4pt}\\[-0.2em]
    %  Niclas Hauerberg Hyldahl (JNC117)\\
    %  {\small \today}
    %\end{minipage}\hfill
    %\begin{minipage}[t]{0.48\textwidth}
    %  \rule{\linewidth}{0.4pt}\\[-0.2em]
    %  Lucas Daniel Paz Zuleta (TZS159)\\
    %  {\small \today}
    %\end{minipage}