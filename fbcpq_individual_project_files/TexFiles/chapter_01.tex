%\chapter*{Preface and abstract}
\setlength{\headheight}{12.71342pt}
\addtolength{\topmargin}{-0.71342pt}


\section*{Preface}
This written report has been prepared as part of the course NPLK14014U - Fruit and Berry Crop Physiology and Quality at the University of Copenhagen. The course explores the physiological and biochemical factors determining fruit development, yield, and quality in relation to fresh consumption and processing. Emphasis is placed on how genetic, environmental, and management factors influence internal and external fruit quality, including the formation of secondary metabolites and sensory attributes such as aroma.

\vspace{1em}
The present project is a theoretical study on the development and composition of aroma in mango, \textit{Mangifera indica L}., The focus is on the biochemical formation of volatile organic compounds (VOCs) and their relationship to fruit ripening and sensory quality. The report integrates physiological and biochemical knowledge to describe how preharvest conditions, cultivar differences, and post-harvest handling affect aroma expression and the perception of fruit quality.

\vspace{1em}
The assignment was carried out by Lucas Daniel Paz Zuleta (TZS159), MSc student at the University of Copenhagen, as an individual written report fulfilling the course requirements for NPLK14014U - Fruit and Berry Crop Physiology and Quality.

\vspace{1.5em}
\section*{Summary}
Mango (\textit{Mangifera indica} L.) is renowned for its distinctive aroma. This characteristic results from a complex mixture of volatile organic compounds (VOCs) that are developed during the fruits ripening and maturation process. These VOCs include terpenes, esters, aldehydes, alcohols, and lactones, which in combination contribute to the mangoes unique sensory qualities and the overall aroma profile. The primary volatile compounds are biosynthesised through three main metabolic pathways: The terpenoid pathway, which produces terpenes. The fatty acid derivative pathway, that forms aldehydes and esters. And the amino acid derivative pathway, where branched chain volatiles are generated. Some key enzymes involved in these pathways are Alcohol Dehydrogenase (ADH), Terpene Synthases (TPSs), and Lipoxygenase (LOX), which during ripening and maturation plays a crucial role in the formation and balance of the aroma profile.
Genetic factors indicates a significant influence on the aroma profile across mango cultivars. New World varieties are generally terpene rich, wheareas Old World cultivars are rich in esters and alcohols. Environmental conditions, as well as post-harvest treatments, has a significant impact on the volatile composition. Ethylene driven ripening enhances aroma formation, while chilling stress will suppress monoterpenes and increase $C_6$/$C_9$ aldehydes. Processing methods, e.g. drying, can lead to significant losses of volatiles, although it has been shown to preserve lactones and esters when osmotic dehydration is applied as a pre-treatment.
Analytical characterisation of volatile compounds in mango relies on solvent-free extraction methods, such as Solid Phase Microextraction (SPME) coupled with Gas Chromatography-Mass Spectrometry (GC-MS). Sensory tools like GC-Olfactometry are also used to identify odour-active compounds. 

\vspace{1em}
Integrating biochemical, environmental, and analytical perspectives provides a foundation for optimizing cultivar selection, post-harvest handling, and good processing practices to obtain a consumer desirable aroma quality in mangoes.



\vspace{2em}
\begin{center}
    \textbf{Signature}\\[0.5em]
    {\small Copenhagen, \today}
    \end{center}
    
    \vspace{2.5em}
    
    \noindent
    \begin{minipage}[t]{0.48\textwidth}
      \stackengine{0pt}{\raisebox{-1.2em}{\includegraphics[width=0.7\linewidth]{KU_titelpage/signatures/lucas_daniel_paz_zuleta.png}}}{\rule{\linewidth}{0.4pt}}{O}{c}{F}{F}{L}\\[0.5em]
      Lucas Daniel Paz Zuleta (TZS159)\\
      {\small \today}
  \end{minipage}\hfill
    %\begin{minipage}[t]{0.48\textwidth}
    %  \rule{\linewidth}{0.4pt}\\[-0.2em]
    %  Nils Hugo Nilsson (XQK212)\\
    %  {\small \today}
    %\end{minipage}
    
    \vspace{7em}
    
    \noindent
    %\begin{minipage}[t]{0.48\textwidth}
    %  \rule{\linewidth}{0.4pt}\\[-0.2em]
    %  Niclas Hauerberg Hyldahl (JNC117)\\
    %  {\small \today}
    %\end{minipage}\hfill
    %\begin{minipage}[t]{0.48\textwidth}
    %  \rule{\linewidth}{0.4pt}\\[-0.2em]
    %  Lucas Daniel Paz Zuleta (TZS159)\\
    %  {\small \today}
    %\end{minipage}

    \newpage