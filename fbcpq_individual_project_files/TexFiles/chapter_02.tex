\setcounter{chapter}{1}
\setcounter{section}{0}
%\chapter{Introduction}
\setlength{\headheight}{12.71342pt}
\addtolength{\topmargin}{-0.71342pt}

\section{Introduction}
Fruits are complex and heterogeneous structures that comprise  a wide variety of metabolites that serve as precursors to volatile organic compounds (VOCs), such as carbohydrates, fatty acids, and pigments \cite*{A03_PanoFarias2017}. The mango fruit is no exception of this complexity and have in recent years been the focus of many studies, studying the activity of volatiles in the fruit \cite*{A04_GUO2023112779}.

\subsection{Background}
Mango is a tropical fruit which belongs to the Anacardiaceae family and is scientifically known as \textit{Mangifera} \cite*{A04_GUO2023112779}. It is popularly characterised as a sweet, juicy, aromatic fruit with a low fibre flesh \cite*{A05_Chin2019}. It is primarily cultivated in tropical and subtropical regions, where it is of significant economic importance \cite*{A05_Chin2019}. 
The annual tropical production of mango is over 46 million tons and is thereby the most produced tropical fruit after banana \cite*{A07_Bonneau2016}. The perishable nature and susceptibility to post-harvest losses and diseases pose challenges for the mango industry, restricting the production and potential \cite*{A05_Chin2019}.

\subsection{Focus on Mango}
The mango is an important fruit crop worldwide, valued for its high nutritional content, with significant levels of fibre, vitamin C and $\beta$-carotene \cite*{A01_Aguirre-Lopez_2023, A07_Bonneau2016}. However, beyond its nutritional benefits, the mango is particularly renowned for its distinctive aroma, which plays a crucial role in consumer preference and marketability \cite*{A06_Badar2016}. s of quality and freshness \cite*{A05_Chin2019}.
A study by Badar et al. (2016) found that aroma is one of the most important quality attributes that influence consumer preference and acceptance of mango fruit \cite*{A06_Badar2016}. The main aroma contributors in mango are the VOCs; aldehydes, alcohols, esters and ketones \cite*{A02_Moreno2010}. Among these, 3-carene, limonene, $\beta$-pinene, acetaldehyde, ethanol and hexanal \cite*{A02_Moreno2010}. Understanding the origin and behaviour of these volatiles is therefore essential for improving fruit quality, post-harvest handling, and processing applications.

\subsection{Aim and Scope}
The aim of this report is to explore the formation, composition, and significance of VOCs in mango fruits. The main focus is on how these compounds contribute to the fruit's aroma and overall quality, both in terms of maintaining freshness and enhancing consumer appeal.
The scope includes an overview of the chemical classes and key VOCs identified in mango, the metabolic- and enzymatic pathways involved in their biosynthesis, and the influence of pre- and post-harvest conditions on their abundance. Finally, the report outlines the methods commonly applied for the extraction and identification of VOCs and their relevance for assessing fruit quality and consumer perception.


\section{Aroma Composition in Mango}
The complexity of aromas in mango is due to the variety of VOCs that are present in the fruit's matrix. These compounds arise from different biochemical pathways and contribute to the overall sensory experience of the fruit \cite*{A05_Chin2019}. These pathways are most prominently manifested during the ripening processes of the fruit. Also visual and textural changes occur during ripening, which further influence the perception of aroma \cite*{A01_Aguirre-Lopez_2023, A05_Chin2019}.

\subsection{Chemical Classes of Mango Aroma Compounds}
The volatile composition in fresh mango fruits has been extensively studied, revealing a diverse array with several hundred identified volatile compounds, occuring in free form in the fruit \cite*{A07_Bonneau2016}. By calculating the flavour dilution factor (FD) of the volatile compounds using gas chromatography-olfactometry (GC-O) analysis, it has been possible to identify the key aroma compounds for the odour-active fraction \cite*{A07_Bonneau2016}.

Mango aroma is primarily composed of several chemical classes. The most dominant classes is monoterpene hydrocarbons, which account for 90.2\% of the total volatile compounds in fresh mango, as illustrated in Figure \ref{fig:mango_aroma_compounds}. Other significant classes include lactones (4.1\%) and sesquiterpene hydrocarbons (3.0\%) \cite*{A07_Bonneau2016}. 

\begin{figure}
    \centering
    \includegraphics[width=0.8\textwidth]{Figures/fig_fresh_mango_chemical_classes.JPG}
    \caption{Distribution $[\%]$ of chemical classes of volatile compounds in fresh mango. Adapted from Bonneau et al. (2016) \cite*{A07_Bonneau2016}.}
    \label{fig:mango_aroma_compounds}
\end{figure}

\subsection{Key Aroma Compounds in Mango}
In this subsection, the key aroma compounds identified from fresh mango fruits in the study by Bonneau et al. (2016) are discussed \cite*{A07_Bonneau2016}. .

\subsubsection*{Terpenes and Terpenoids}
Terpenes represent the largest class of mango aroma compounds, derived from isoprene units via the terpenoid biosynthetic pathway \cite*{A09_Barras2024}. They include both monoterpenes and sesquiterpenes, which together account for the majority of mango volatiles \cite*{A07_Bonneau2016}. Terpenes and terpenoids are found in many different natural sources, including fruits, plants, animals, microbes, and fungi. The terpenes belong to the largest class of secondary metabolites in nature and consist of five connected carbon atoms, known as isoprene units. These carbon units can be assembled in thousands of ways \cite*{B01_TerpenesTerpenoids_2018}. The terpenoids are further subcualified into five sub-groups based on the number of isoprene units they contain: monoterpenes (C10), sesquiterpenes (C15), diterpenes (C20), sesterterpenes (C25), and triterpenes (C30) \cite*{B01_TerpenesTerpenoids_2018}.


\paragraph*{Monoterpene hydrocarbons}
A total of 11 monoterpene hydrocarbons were identified as key aroma compounds in fresh mango \cite*{A07_Bonneau2016}. The most significant ones include: $\alpha$-phellandrene, $\gamma$-terpinene, $\delta$-3-carnene, $\beta$-myrcene, $\alpha$-terpinene, limonene, $\beta$-phellandrene, and $\alpha$-terpineol. 

Monoterpenes are the smallest molecules in the isoprenoid family with conserved hydrocarbons \cite*{A09_Barras2024}. They share the formula $C_{10}H_{16}$ and over 400 different chemical structures has been classified as such \cite*{A09_Barras2024}. The key monoterpene hydrocarbons identified in mangos are reported to have significant impact on the overall odorants \cite*{A07_Bonneau2016}.

\paragraph*{Sesquiterpene hydrocarbons}
Compared to monoterpenes, sesquiterpenes are larger molecules with the formula $C_{15}H_{24}$. In mango fruit, the study by Bonneau et al. (2016) identified four key sesquiterpene hydrocarbons, including: $\alpha$-gurjunene, $\alpha$-copaene, $\beta$-caryophyllene, and $\alpha$-caryophyllene \cite*{A07_Bonneau2016}.

\subsubsection*{Alcohols and Aromatic Alcohols}
Alcohols and aromatic alcohols are, like terpenes and terpenoids, major contributors to the key aroma of mango, though they differ in chemical structure and biosynthetic origin. A study by Singh et al. (2010) highlighted the role of alcohol dehydrogenase (ADH) in the enzymatic reduction of aldehydes leading to the formation of these compounds \cite*{A10_Singh2010}.

\vspace{1em}
A total of nine alcohols were identified as key aroma compounds in fresh mango \cite*{A07_Bonneau2016}. The most significant ones include: 2-methyl-1-propanol, 1-pentanol, (\textit{E})-2-penten-1-ol, (\textit{Z})-2-penten-1-ol, 1-octanol, 1-butanol, 3-methyl-1-butanol, 2-decanol, 1-hexanol, and (\textit{Z})-3-hexen-1-ol \cite*{A07_Bonneau2016}.

\subsubsection*{Aldehydes and Aromatic Aldehydes}
Aldehydes are volatile compounds primarily formed through the oxidation of unsaturated fatty acids, such as $\alpha$-linolenic acid, via the lipoxygenase (LOX)-hydroperoxide lyase (HPL) pathway. These reactions generate C$_6$ and C$_9$ aldehydes, often referred to as green leaf volatiles, which are associated with fresh and grassy notes in the mango fruit aroma \cite*{A11_Sivankalyani2017}. In mango, this pathway becomes particularly active under chilling stress, leading to increased levels of compounds such as 1-hexanal, (\textit{E})-2-hexenal, and (\textit{Z})-3-hexenal before visible quality loss occurs.

\vspace{1em}
In fresh mango, a total of six aldehydes were identified as key aroma compounds \cite*{A07_Bonneau2016}. The most significant ones include hexanal, (\textit{E,E})-2,4-heptadienal, nonanal, and (\textit{E,Z})-2,4-heptadienal \cite*{A07_Bonneau2016}.

\subsubsection*{Lactones}
Lactones are VOCs that quantitatively represents a smaller fraction of the mango aroma profile compared to terpenes and terpenoids. They are nonetheless important contributors to the overall sensory experience of the fruit, and make up two times the quantitative amount of alcohols, ketones, esters, and furans combined \cite*{A14_Silva2021, A07_Bonneau2016}. Lactones are cyclic esters formed through the intramolecular esterification of hydroxy acids. In mango, they are primarily derived from the oxidation and subsequent cyclization of fatty acids during the ripening process \cite*{A13_ElHadi2013}.

\vspace{1em}
A total of five lactones were identified as key aroma compounds in fresh mango, in the study by \textcite*{A07_Bonneau2016}. The most significant ones include: $\alpha$-methyl-$\gamma$-bytyrolactone, $\gamma$-hexalactone, $\delta$-hexalactone, and $\delta$-ocatlacetone \cite*{A07_Bonneau2016}.

\subsubsection*{Minor Volatile Compounds: Ketones, Esters, and Furans}
Ketones, esters, and furans are present in much smaller amounts compared to terpenes and terpenoids, but still contribute to the complexity of mango aroma \cite*{A07_Bonneau2016}. Ketones can arise from the enzymatic reduction of carbonyl compounds, as demonstrated by bioreduction studies on tropical fruit tissues \cite*{A12_Lemos2008}. Esters are associated with fruity and sweet nuances, whereas the furans provide the mango fruit with mild caramel-like notes. Despite low abundance of these three classes of compounds, they may enhance the overall balance of aroma perception in fresh mango \cite*{A13_ElHadi2013}.

\subsection{Variation in Aroma Profiles Among Mango Varieties}
The volatile composition of mango (\textit{Mangifera indica} L.) varies substantially among cultivars, reflecting genetic differences and environmental influences on fruit metabolism and ripening \cite*{A01_Aguirre-Lopez_2023}. Comparative analyses of multiple mango varieties has shown that the abundance and composition of key aroma compounds differ significantly across cultivars, leading to distinct sensory profiles \cite*{A01_Aguirre-Lopez_2023,A02_Moreno2010}. 

\vspace{1em}
A recent study by \textcite{A16_Tandel2023} investigated 16 different Indian mango cultivars, revealing significant differences in their volatile profiles \cite*{A16_Tandel2023}. Cultivars such as \textit{Alphonso}, \textit{Kesar}, and \textit{Ratna}, were characterised by their high levels of terpenes, whereas the variety \textit{Amrapali} was the dominant in esters \cite*{A16_Tandel2023}. The cultivar \textit{Kesar} exhibited the highest content of monoterpenes, showcasing a total of 91.00\%, while two other cultivars, \textit{Dashehari} and \textit{Neelum} had 0.00\% monoterpenes \cite*{A16_Tandel2023}. Instead, the \textit{Dashehari} cultivar were rich in sesquiterpenes and fatty acids, showing a total of 23.76\% and 37.91\%, respectively \cite*{A16_Tandel2023}.

Within the terpene class, \textit{allo-ociemene} was the dominant monterpene across most cultivars, with a concentration ranging from 4.99\% in \textit{Neeleswari} to 89.36\% in \textit{Sonpari} \cite*{A16_Tandel2023}. The cultivar \textit{Amrapali} stood out with a high ester content of 18.49\%, primarily due to the presence of butanoic acid \cite*{A16_Tandel2023}.

\vspace{1em}
A study by \textcite{A13_ElHadi2013} further distinguished the aroma profiles of different mango cultivars according to their geographical origin \cite*{A13_ElHadi2013}. New Worlds varieties, e.g. \textit{Haden}, \textit{Irwin}, \textit{Manila}, and \textit{Tommy Atkins}, were typically dominated by terpene hydrocarbons—especially $\delta$-3-carene. 

Especially 3-carene was a dominant compound in New World mangoes, and constituted 16-90\% of total volatiles, followed by  $\alpha$-pinene, $\alpha$-phellandrene, and $\sigma$-3-carene \cite*{A13_ElHadi2013}. In contrast, Old World varieties tended to exhibit higher proportions of esters, alcohols, and ketones, resulting in sweeter and fruitier aroma profiles \cite*{A13_ElHadi2013}, althrough, \textcite{A13_ElHadi2013} reported a larger difference in VOC quantity and quality in Old World cultivars compared to New World ones \cite*{A13_ElHadi2013}. 

\vspace{1em}
As discussed by \textcite{A16_Tandel2023}, previous studies have compared two Chinese mango cultivars, \textit{Tainong} and \textit{Jinmang}, which exhibited high levels of $\beta$-ocimene, myrcene, and $\alpha$-terpinolene, which significantly contributed to their aroma profiles. These observations align with the findings of \textcite{A15_Xie2023}, who also reported terpenes as being a dominant volatile class in the Chinese cultivar \textit{Jinmang} \cite*{A16_Tandel2023, A15_Xie2023}.


\vspace{1em}
\textcite{A15_Xie2023} conducted a detailed comparison between the Chinese cultivars \textit{Tainong} and \textit{Hongyu}, revealing cultivar-specific differences in their volatile composition \cite*{A15_Xie2023}. The \textit{Tainong} mango contained a higher concentration of terpenes and aldehydes, thereby producing a characteristic grassy aroma.  In contrast, the \textit{Hongyu} exhibited higher ester levels, contributing to a more fruity and sweet aroma \cite*{A15_Xie2023}. Key distinguishing VOCs included $\beta$-ocimene for the terpene-lemon profile of \textit{Tainong} and propyl butyrate for the fruity character of \textit{Hongyu} \cite*{A15_Xie2023}.

\section{Biochemical Pathways of Aroma Compound Formation}
Volatile organic compounds (VOCs) are low-molecular-weight molecules comprising functional groups such as alcohols, esters, ketones, aldehydes, and terpenes \cite*{A01_Aguirre-Lopez_2023,B01_TerpenesTerpenoids_2018}. In fruits, VOCs originate from metabolic and enzymatic reactions during maturation and ripening, shaping the fruit’s characteristic aroma profile \cite*{A01_Aguirre-Lopez_2023}. VOC profiling is therefore a key indicator of fruit quality, cultivar differentiation, and ripeness \cite*{A01_Aguirre-Lopez_2023}. The mango aroma profile mainly consists of esters (34.0\%), aldehydes (24.2\%), alcohols (16.5\%), and other volatiles such as terpenes, ketones, lactones, and furans (25.3\%) \cite*{A01_Aguirre-Lopez_2023}.

\vspace{1em}
The study of these mechanisms, known as \textit{volatilomics}, investigates the biogenesis and metabolic routes of aroma compounds \cite*{A01_Aguirre-Lopez_2023}. Their biosynthesis is based on three major precursor classes—amino acids, fatty acids, and carbohydrate- or terpenoid-derived intermediates—which give rise to the major biosynthetic pathways described below \cite*{A13_ElHadi2013}. Once the basic molecular frameworks are formed, enzymatic transformations such as hydroxylation, acylation, methylation, oxidation-reduction, and ring closure enhance volatility and molecular diversity \cite*{A13_ElHadi2013}.

\subsection{Terpenoid Pathway (MEP and MVA Pathways)}
Terpenoids, including monoterpenes ($C_{10}$) and sesquiterpenes ($C_{15}$), constitute the largest class of plant secondary metabolites \cite*{B01_TerpenesTerpenoids_2018}. All derive from the $C_{5}$ precursors isopentenyl diphosphate (IPP) and dimethylallyl diphosphate (DMAPP), synthesized through two independent routes: the cytosolic mevalonic acid (MVA) pathway and the plastidial methylerythritol-4-phosphate (MEP) pathway \cite*{A13_ElHadi2013}. The MVA pathway begins with acetyl-CoA condensation, while the MEP pathway utilizes pyruvate and glyceraldehyde-3-phosphate as substrates \cite*{A09_Barras2024}. Cross-talk between both routes enables exchange of intermediates, broadening terpene diversity. Subsequent elongation by prenyltransferases and cyclization by terpene synthases (TPSs) generate the characteristic monoterpenes and sesquiterpenes of mango aroma \cite*{A09_Barras2024,A13_ElHadi2013}.

\subsection{Fatty Acid Derivative Pathway}
Fatty acids from membrane lipids serve as precursors of aldehydes, alcohols, esters, and lactones. Two main catabolic routes are involved \cite*{A05_Chin2019,A13_ElHadi2013}: (i) the $\beta$-oxidation pathway, which removes successive $C_{2}$ units (acetyl-CoA) to yield shorter acyl-CoAs later reduced by alcohol dehydrogenases (ADH) and esterified by alcohol acyltransferases (AAT); and (ii) the lipoxygenase (LOX) pathway, which converts linoleic and linolenic acids into $C_{6}$ and $C_{9}$ aldehydes and alcohols, known as green leaf volatiles. The $\beta$-oxidation route also contributes to lactone formation through chain shortening of hydroxylated fatty acids followed by spontaneous lactonization \cite*{A14_Silva2021}.

\subsection{Amino Acid Derivative Pathway}
Branched-chain amino acids—leucine, isoleucine, and valine—act as precursors for branched-chain alcohols, aldehydes, and esters \cite*{A13_ElHadi2013}. These undergo transamination to form $\alpha$-keto acids, followed by decarboxylation and subsequent reduction or esterification reactions to yield the volatile end-products. Such compounds often impart fruity or fermented notes in ripe mango.

\vspace{1em}
Regardless of precursor origin, the final aroma diversity arises from secondary enzymatic modifications that refine the volatility and structure of the compounds produced \cite*{A13_ElHadi2013}.

\subsection{Enzymes Involved in Aroma Biosynthesis}
The diversity of volatile organic compounds (VOCs) in mango arises from enzymatic modifications such as hydroxylation, acylation, methylation, oxidation-reduction, and ring closure \cite*{A13_ElHadi2013}. Alcohol dehydrogenases (ADH; EC 1.1.1.1) play a key role in converting aldehydes to alcohols and supplying substrates for ester synthesis. Mango expresses multiple ADH isoforms (MiADH1-3) that are differentially regulated during ripening \cite*{A10_Singh2010}. Other enzymes contributing to aroma formation include methionine $\gamma$-lyase and 3-ketoacyl-CoA thiolase B, active in amino-acid- and fatty-acid-derived pathways \cite*{A05_Chin2019}. In the fatty acid route, lipoxygenase (LOX) and hydroperoxide lyase (HPL) cooperate with ADH to produce $C_6$ and $C_9$ volatiles, while terpene synthases (TPSs) catalyse the cyclisation of prenyl diphosphate precursors in the terpenoid pathway \cite*{A13_ElHadi2013}.

\subsection{Terpenoid Pathway (MEP and MVA Pathways)}
Terpenoids, including the monoterpenes ($C_{10}$) and sesquiterpenes ($C_{15}$) prominent in mango aroma (e.g., $\beta$-pinene, $\beta$-caryophyllene, germacrene D), represent the largest class of plant secondary metabolites \cite*{A02_Moreno2010,A04_GUO2023112779,A09_Barras2024,A13_ElHadi2013}. All share the $C_{5}H_{8}$ isoprenoid unit and derive from the universal precursors isopentenyl diphosphate (IPP) and dimethylallyl diphosphate (DMAPP), synthesized via two spatially distinct routes: the cytosolic mevalonic acid (MVA) pathway, starting from acetyl-CoA, and the plastidial methylerythritol-4-phosphate (MEP) pathway, using pyruvate and glyceraldehyde-3-phosphate (G3P) \cite*{A09_Barras2024,A13_ElHadi2013}. Cross-talk between these pathways enables exchange of intermediates. Subsequent elongation of IPP and DMAPP by prenyltransferases produces geranyl (GPP) and farnesyl diphosphates (FPP), which terpene synthases (TPSs) cyclize into the diverse terpene skeletons that define mango aroma \cite*{B01_TerpenesTerpenoids_2018,A13_ElHadi2013}.


\subsection{Fatty Acid Derivative Pathway}

\subsection{Amino Acid Derivative Pathway}


\section{Environmental and Genetic Factors Influencing Aroma Production}
\subsection{Pre-harvest Factors}
\subsection{Post-harvest Factors}
\subsection{Processing Implications on Aroma Retention}


\section{Analytical Techniques for Aroma Compound Identification}
\subsection{Extraction and Analysis Methods}
\subsection{Quantification and Sensory Evaluation}

\section{Comparative Perspective}

\section{Conclusion}
