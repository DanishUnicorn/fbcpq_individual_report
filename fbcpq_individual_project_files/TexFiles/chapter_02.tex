\setcounter{chapter}{1}
\setcounter{section}{0}
%\chapter{Introduction}
\setlength{\headheight}{12.71342pt}
\addtolength{\topmargin}{-0.71342pt}

\section{Introduction}
Fruits are complex and heterogeneous structures that comprise  a wide variety of metabolites that serve as precursors to volatile organic compounds (VOCs), such as carbohydrates, fatty acids, and pigments \cite*{A03_PanoFarias2017}. The mango fruit is no exception of this complexity and have in recent years been the focus of many studies, studying the activity of volatiles in the fruit \cite*{A04_GUO2023112779}.


\subsection{Background}
Mango is a tropical fruit which belongs to the Anacardiaceae family and is scientifically known as \textit{Mangifera} \cite*{A04_GUO2023112779}. It is popularly characterised as a sweet, juicy, aromatic fruit with a low fibre flesh \cite*{A05_Chin2019}. It is primarily cultivated in tropical and subtropical regions, where it is of significant economic importance \cite*{A05_Chin2019}. 
The annual tropical production of mango is over 46 million tons, but is restricted by its perishable nature and susceptibility to post-harvest losses \cite*{A05_Chin2019}.

\subsection{Focus on Mango}
A study by Badar et al. (2016) found that aroma is one of the most important quality attributes that influence consumer preference and acceptance of mango fruit \cite*{A06_Badar2016}. The main aroma contributors in mango are the VOCs; aldehydes, alcohols, esters and ketones \cite*{A02_Moreno2010}. Among these, 3-carene, limonene, $\beta$-pinene, acetaldehyde, ethanol and hexanal \cite*{A02_Moreno2010}. Understanding the origin and behaviour of these volatiles is therefore essential for improving fruit quality, post-harvest handling, and processing applications.


\subsection{Aim and Scope}
The aim of this report is to explore the formation, composition, and significance of VOCs in mango fruits. The main focus is on how these compounds contribute to the fruit's aroma and overall quality, both in terms of maintaining freshness and enhancing consumer appeal.
The scope includes an overview of the chemical classes and key VOCs identified in mango, the metabolic- and enzymatic pathways involved in their biosynthesis, and the influence of pre- and post-harvest conditions on their abundance. Finally, the report outlines the methods commonly applied for the extraction and identification of VOCs and their relevance for assessing fruit quality and consumer perception.


\section{Aroma Composition in Mango}
\subsection{Chemical Classes of Mango Aroma Compounds}
\subsection{Key Aroma Compounds in Mango}
\subsection{Variation in Aroma Profiles Among Mango Varieties}


\section{Biochemical Pathways of Aroma Compound Formation}
\subsection{Overview of Biosynthetic Pathways}
\subsection{Enzymes Involved in Aroma Biosynthesis}
\subsection{Terpenoid Pathway (MEP and MVA Pathways)}
\subsection{Fatty Acid Derivative Pathway}
\subsection{Amino Acid Derivative Pathway}


\section{Environmental and Genetic Factors Influencing Aroma Production}
\subsection{Pre-harvest Factors}
\subsection{Post-harvest Factors}
\subsection{Processing Implications on Aroma Retention}


\section{Analytical Techniques for Aroma Compound Identification}
\subsection{Extraction and Analysis Methods}
\subsection{Quantification and Sensory Evaluation}

\section{Comparative Perspective}

\section{Conclusion}
